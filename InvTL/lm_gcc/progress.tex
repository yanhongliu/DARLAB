\documentclass[11pt,letterpaper]{article}

\title{INVTL/C Project Outline}

\section{Goal of the Project}
The aim of the project is to let the programmer write easy instrumentation or 
transformation code for C, using rules. Without a rule-based language to do this 
in, it is a very time-complex and error-prone task.

Our goal is to have a system, that, when given set of InvTL rules, and a subject 
program (in C) to be transformed, will perform the necessary static analysis of 
the subject program, and transform it in accordance to the InvTL rules.

For example, we wish to print out when files are opened for future fprintfs.
The following short rule would accomplish that task.

\begin<verbatim>
inv $r=C{fprintf($fd,$params)}
at C{$fd=fopen($fname,$mode)}
do before C{
    printf("Opened $fname for future write at $r");
}
\end{verbatim}

\section{Project Implementation Overview}
*******************************************************************
Terminology
*******************************************************************
GCC:       The GCC 4.1.2 compiler, with the C/C++ modules
Python:    Python 2.4
ELSA:      Elsa-derived C++ parser, with support for patterns
*******************************************************************

The project boils down to designing and implementing InvTL/C, a C
transformation module for InvTS. 

Rule application is done in several steps, in the following sequence:
    A) Rules are read in and analyzed
    B) Subject program is read in and analyzed
    C) Rules are applied, taking into account the results of B.

Due to the structure of InvTS described prior, we implement the C parsing, 
pattern matching, analysis, and replacement as a language module for InvTS.

To implement this module, we would have to accomplish the following tasks:
    1. Parse the patterns, which are incomplete C code
    2. Parse subject C code (Which may be very complex, and must take into 
        account preprocessing)
    3. Perform matching of patterns against the subject program.
    4. Perform static analysis of subject program for application of if-clauses, 
        and update site detection.
    5. Determine update sites.
    6. Perform replacements on the subject code where necessary 
        (at update sites).

Step 1 requires a C parser that can parse incomplete C statements, with
wildcards. As the GNU C grammar is context and type sensitive, developing this
parser from scratch is unfeasible. ELSA was chosen as a basis, and augmented so
that it emits alternatives where multiple interpretations due to lacking
context exist. We call the module responsible for this the C Parser & Query
Generation Module (C->GIMPLE). The reason to not use the GCC parser and
framework (such as GCC-XML) is that GCC cannot compile incomplete C, as it
type-checks during the compilation phase.

Steps 2-6 require knowledge of how the program is preprocessed, the order in
which it is compiled, etc... This, coupled with the complexity of doing type
checking and static analysis for GNU C, made us decide to use the GCC
infrastructure itself. To this end, there are 2 programs that perform steps
2-5.  The 1st program is a GCC plugin (We have extended GCC with a plugin
architecture) which has access to the GCC internals, and which exposes a
predefined interface to the GCC LM written in Python. This part is called the
Python Plugin.  Step 2 is thus performed entirely by GCC, as the subject
program is presumably complete, and compilable prior to the transformation.

Step 3 involves matching 2 ASTs, one of which contains wildcards and
alternatives. The 1st part of this step (taking a AST of a pattern, and
converting it to a representation that is easily matched to GIMPLE) is
performed by the Query Generation Module (C->GIMPLE), which is written in
Python. The query is then executed by the backend, which is a module in the GCC
plugin.  This architecture allows us to write complex queries while increasing
the complexity of only the GCC LM, which is written in Python. For example, we
can integrate arbitrary parts of if clauses into the query. The backend (in C)
is called the Query Module, while the front-end  (in Python) is called the
Query Generation Module.

Step 4 is performed entirely in GCC, using the analysis infrastructure 
available. The results of the analysis are used to annotate the GIMPLE 
representation. This results GIMPLE code looking like:
    $a=b {aliases:{...},cfgnode_prior:_some_node_, ....}$
The GCC plugin exposes an API to query for annotations of GIMPLE statements. 
This API is used by the GCC LM to implement the functions available in the if 
and de clauses. This part is called the Analysis Module.

Step 5 is performed by the GCC LM, which uses the GCC plugin API to query all
syntactic matches. This mainly involves querying the annotations on the matched
GIMPLE expressions. As this step becomes more and more sophisticated, its
implementation in Python (and not in C) pays off more and more, due to the
development speed differential.

Step 6 is performed by the GCC plugin, whose API function to replace code are
called (without much complicated logic, once update sites are located and
metavariables bound) are called by the GCC LM. This module is called the Rule
Application Module.

After this point the rule has been applied, the GCC Plugin relinquishes control
to GCC, and GCC continues to compile the program.


\section{Architecture Overview}
=============================================================================
Architecture of InvTS/GCC
1. Overall Architecture:
	GCC (The Backend)
		Plugin Architecture
		Python Plugin
		Analysis Module
		Query Module (Pattern matching backend)
		Rule Application Module 
	InvTS (The Frontend)
		C Parser & Query Generation Module (C->GIMPLE)
		Inter-invocation module (via database)
		InvTL LM
	Both
		Various LM/GCC Functions

Description of individual parts:
GCC - A Plugin Architecture:
	This module uses the ability to load plug-ins (written in C) into GCC.
	A plugin, once loaded, has functions executed at:
		Beginning of translation of a translation unit (c file).
		Beginning of translation of each function.
		End of translation of each translation unit.
    The plugin has access to all compile-time and run-time information available
    to GCC during the GIMPLE optimization phase.  It acts as a normal optimization
    and can perform arbitrary modifications on a per-function basis.

GCC - Python Plugin
	This particular plugin has the Python binary linked into it, allowing it to 
	execute arbitrary Python programs within the translation phase of GCC.
	It also exports a Python module (GCC) which allows the Python program to 
	call functions exposed by the plugin, such as match, and others. 
    The plugin passes user-specified arguments to a linked-in Python script that
    in turn locates and invokes the InvTS core and provides those arguments to it.	
    
GCC - Analysis Module:
    This module forms a part of the interface provided by the GCC plugin. Its 
    main purpose is to 	maintain a map from each GIMPLE statement to a database 
    of analysis data that can later be queried by the Python plugin. There are 
    several analyses it needs to perform:
		Control-flow analysis
		Local Alias Analysis (Intraprocedural)
		Global Alias Analysis (Interprocedural)
		Data-flow analysis
		.. and some others (To be determined, as needed)

GCC - Query Module (Pattern matching frontend/backend)
	This module mainly exports a function find_locs(query, metavariable_bindings).
	This function returns a set of tuples, consisting of 
	(opaque object, delta-bindings), where the "opaque object" is some sort of 
	reference to the GIMPLE construct that was matched by the query with the 
	bindings, and delta-bindings is the set of modified bindings due to the 
	match.
	_query_ has the format of a list of tuples, where each tuple consists of the 
	GIMPLE construct to match, the compile-time evaluable constraints in CNR 
	form, and a callback function that is called by GIMPLE to determine whether 
	the match is successful. The GIMPLE construct is stored as a tree. 
	The constraints are stored as a list of of lists, in prefix notation. 
	Overall, it looks like: 
	find_locs((
		(GIMPLE_TREE,( (is_integer,blah_variable), (is_float,t_variable)),None),
		(GIMPLE_TREE, ...),
		...
	)
	The order of the trees in the list is the order in which constructs must 
	occur in the target code to be matched.

GCC - Rule Application Module 
	This module exports 3 functions: 
		apply_before(location,code)
		apply_after(location,code)
		apply_instead(location,code)
	We describe the apply_before function, as the other are similar.
	The function inserts the code before location, and then lets GCC proceed 
	with the compilation.
	location is the "opaque object" returned by query.
	code is the C code, with all metavariables bound and substituted.
	If possible, the module will compile code separately, and then insert the 
	resulting GIMPLE code (before/after/instead of) location. If not possible,
	the function is recompiled, with the code textually inserted.

InvTS - C Parser
	The parser is based on ELSA. Its main goal is to take a C expression, and to
	generate a (preferably typed) AST that can be converted into a query for 
	_query_. As the argument to query is not a GIMPLE AST, but an object 
	resembling the AST, and as GCC cannot even parse truly incomplete code, the
	parser is independent of GCC, and generates its own AST and typing rules.
	The parser then performs typing according to C typing rules, and infers 
	concrete types, where possible. If concrete types are impossible to infer, 
	but the type equations are not degenerate, the parser infers typing rules 
	via deduced types. For example, a.x+1 infers that a.x<int|long|float|...>
	where ... is all the types to which one can add a one.
	Further, elaboration is done where possible, thus, types are not just a<int>, 
	but a<int[1,2]> , which means that a is an int with values 1 and 2. Clearly,
	it is not always possible to infer such things, and at that point, the 
	parser drops back to inferring just types. 

InvTS - Query Generation Module	 
	This AST is then converted into the query form by a slicing procedure, 
	followed by a transformation done via a traversal. The slicing procedure 
	removes the parts of the generated AST that are irrelevant to the query, 
	but needed for typing and semantic elaboration.	The transformation is just 
	a single pass that renames and reorders AST nodes to better match what can 
	be easily queried over GIMPLE ASTs. Metavariables are replaced by wildcards 
	with bindings, and ambiguities are expanded into multiple queries.
	
InvTS - Inter-invocation module (via database)
	The need for an inter-invocation module is exemplified by the following 
	compilation procedure (pseudo-code):
		gcc a.c -output a.o
		gcc b.c -output b.o
		link a.o b.o
	This means that during the invocation of InvTS for a.c, it does not know 
	about b.c.
	This presents a problem if an at clause for an inv matched in a.c is matched 
	in b.c.
	The solution is to maintain state in the database, and require 2 invocations
	of InvTS. The first invocation computes all the inv sites (and the bindings 
	at them), and the 2nd invocation then actually computes the at sites, and 
	performs the transformation. 
	The required magic to get make to understand this is still to in 
	development. The main idea, though, is to set the date of the to file after 
	the first invocation to the past, so that make will be forced to rebuild it 
	for the 2nd invocation.
	
InvTS - InvTL LM
	The LM handles the interaction between the GCC Python plugin and InvTL. It 
	is implemented according to the spec in the docs (Manual/Python LM.pdf).

Various LM/GCC Functions	
	These functions must be implemented in the backend, for the InvTL GCC LM to
	query them when needed. (Usually, for the evaluation of either the inv, at 
	or if clauses of the rules).
	incallgraph(p1,p2)		Determines if p2 can be reached from p1.
	aliased_local(p1,p2)	Determines if p2 is locally the same as p1.
	prior(p1,p2)			Determines if p2 occurs prior to p1, in a single function.
	after(p1,p2)			Determines if p2 occurs after to p1, in a single function.
	reachable(p1,p2)		Determines if p2 is reachable from p1, in a single function.
	correct(p1)
	istype(p1,p2|type-expr)	Determines if p1 has the same type as p2.
	isclass(p1,{numeric|literal|var|ref|function|type})
							Self-explanatory
	must_aliased(p1,p2)		Self-explanatory
	may_aliased(p1,p2)		Self-explanatory


\section{GCC - A Plugin Architecture}
\TODO{Insert Sean's description here}
 
\section{A Python Plugin for GCC}
\TODO{Insert Sean's plugin description here}

\section{Static Analysis for C in GCC}
For precise update detection, we need certain analyses of C code. These analysis 
include: 
\begin{list}
	\item Control-flow analysis
	\item Data-flow analysis
	\item Interprocedural alias analysis
	\item Intraprocedural alias analysis
\end{list}
The resultant information is stored in separate data structures to which each
GIMPLE node has a reference.

\subsection{Control Flow Analysis}
Control flow data is used in several places inside InvTS:
\begin{list}
	\item Refinement of interprocedural alias analysis.
	\item Localization of do before and do after locations.
	\item Localization of de locations.
	\item Reduction of search space for matching conditional patterns.
	\item Various helper functions from the LM depend on it.
\end{list}

\paragraph{Algorithm.}
Control flow analysis of C code is relatively straightforward, with a single
caveat: function pointers.  The algorithm used is rapid type analysis
\cite{Bacon and Sweeney; OOPSLA�96}, which resolves about 80\% of all
function-pointer-based calls correctly, due to type signature analysis. We have
extended this analysis to use aliasing information that is obtained by local
and global alias analysis.

The algorithm first runs without aliasing information, as it's output is used
in the alias analysis algorithms.

Once the alias analysis algorithms have ran, the results are used to re-run
control flow analysis, with aliasing information. This resolves up to 93\% of
function calls that are due to function pointers. This refined control-flow
graph is then fed into interprocedural alias analysis. This results in smaller
may-alias sets.

The analysis time for each run is O(N*Log(N)\^3), N being the number of basic
blocks in the program being analyzed.

\paragraph{Representation.}
The results of CFA are represented as a graph, with each GIMPLE statement
containing a pointer to the node of the CFG which it is contained in. Thus,
each GIMPLE statement annotation contains a key ``CFG_NODE'', the value of
which is the pointer to the proper CFG node.

\subsection{Local Alias Analysis - Intraprocedural Alias Analysis}
Local alias analysis is required as GIMPLE introduces new variables that are
used as temporaries in complex C expressions. 

\paragraph{Usage.}
Local aliasing information is used in 2 places in the program:
\begin{list}
	\item Determining update locations.
	\item Output used in interprocedural (global) alias analysis
\end{list}

\paragraph{Algorithm.}
As functions are small, we use a variant of Choi's \cite{choi99} analysis,
which is $O(N^3)$. The algorithm is context and flow-sensitive. We extend the
algorithm to include input and output quantifiers. The idea of input
quantifiers was taken from \cite{Henkel05}, and was extended by us to produce
output quantifiers.

As the algorithm depends on control-flow information, we run CFA prior to it,
and use the results in computing may-alias sets. Unfortunately, the CFA is not
100\% precise, which results alias sets that are too conservative. To handle
this, we use iterative passes of (CFA,Local Alias Analysis, Global Alias
Analysis). Theoretically, one could run the above triple until a fixed point or
cycle is reached, but this is too time-consuming in practice. We choose to do
2 passes of the algorithms, which empirically gives us good results
\TODO{Experiments}, and an acceptable running time.

\subsection{Global Alias Analysis - Interprocedural Alias Analysis} 
\TODO{A Work in Progress}
We use a cloning-based analysis that uses bdds as representation of contexts.
Running time is currently $O(N^(3+Log(N)))$ for N being the number of basic
blocks in the program. 
\TODO{This is TOO SLOW, fix this}


\section{Query Engine - A Query Engine for GIMPLE}
The GIMPLE Query engine (GQE) is called by the query frontend of InvTS/GCC. It
evaluates queries, and returns matching GIMPLE subtrees as a list.

This module mainly exports a function find_locts.  This function returns a set
of tuples, consisting of (opaque object, delta-bindings), where the "opaque
object" is some sort of reference to the GIMPLE construct that was matched by
the query with the bindings, and delta-bindings is the set of modified bindings
due to the match.  _query_ has the format of a list of tuples, where each tuple
consists of the GIMPLE construct to match, the compile-time evaluable
constraints in CNR form, and a callback function that is called by GIMPLE to
determine whether the match is successful. The GIMPLE construct is stored as a
tree.  The constraints are stored as a list of of lists, in prefix notation. 

Overall, it looks like: 
find_locs((
	(GIMPLE_TREE,( (is_integer,blah_variable), (is_float,t_variable)),None),
	(GIMPLE_TREE, ...),
	...
)
The order of the trees in the list is the order in which constructs must occur
in the target code to be matched.


\subsection{API.}
The GQE implements a simple API consisting of 2 functions:
    
def find_locs(query,boundvars)
	"Unconstrained search. See find_locs_constrained"
def find_locs_constrained(query,scope,boundvars)
        "Given a query, and bound variables in it, returns all matches in scope"

\subparagraph{Parameters.}
query: a tuple, consisting of a construct based on patterns, a condition list, and a
callback function.

construct: See Semantics paragraph

pattern: expressed as a nested dict.
\verb| $type a; | 
would look like:
\begin{verbatim}
	spec : TS_name:
	  name : PQ_name:
		name : "$type"
	  typenameUsed : false
	decllist: 1
	  0 : Declarator
		context : DC_UNKNOWN
		decl : D_name
		  name : PQ_name
			name : "a"
		init : null
		ctorStatement : null
		dtorStatement : null     
\end{verbatim}
	

\verb| $c.$d($c->x,12) |
would look like:
\begin{verbatim}
	spec : TS_simple:
	  id : <implicit-int>
	decllist: 1
	  0 : Declarator
		context : DC_UNKNOWN
		decl : D_name
		  name : PQ_name
			name : "blah"
		init : IN_expr
		  e : E_funCall
			func : E_fieldAcc
			  field: null
			  obj : E_variable
				var: null
				name : PQ_name
				  name : "$c"
			  fieldName : PQ_name
				name : "$d"
			args: 2
			  0 : ArgExpression
				expr : E_arrow
				  obj : E_variable
					var: null
					name : PQ_name
					  name : "$c"
				  fieldName : PQ_name
					name : "x"
			  1 : ArgExpression
				expr : E_intLit
				  i: 0
				  text : "12"
			retObj : null
		ctorStatement : null
		dtorStatement : null     
\end{verbatim}

condition-list: A list of conditions, in CNR form.

callback function: A function pointer to a python function to be called on each
potential match.

boundvars: a dict with keys being variables (strings), and values being 
match-objects. (See below)

scope: either a string constant "file", or "function", or a scope returned 
as a match. (See below)

\subparagraph{Return Values}:
Matches are returned as lists of tuples: 
	(Scope-object,match-object,new-boundvars)
Scope-object: opaque, but with the requirement that:
	find_locs_constrained(pattern,boundvars,scope-object) 
		subset 
	find_locs_constrained(pattern,boundvars,original-scope)
Match-object: entirely opaque.
new-boundvars: a dict with keys being variables (strings), and values being 
match-objects.
Scope and match-object are of instances of class gcc.GIMPLE.

\paragraph{GIMPLE Objects}
Valid GIMPLE objects are defined by the following:

\begin{verbatim}
#define ALL_EXPRS	COMPONENT_REF, BIT_FIELD_REF, INDIRECT_REF, ALIGN_INDIRECT_REF, MISALIGNED_INDIRECT_REF, ARRAY_REF, ARRAY_RANGE_REF, OBJ_TYPE_REF, \
			EXC_PTR_EXPR, FILTER_EXPR, CONSTRUCTOR, COMPOUND_EXPR, MODIFY_EXPR, INIT_EXPR, TARGET_EXPR, COND_EXPR, VEC_COND_EXPR, BIND_EXPR, CALL_EXPR, WITH_CLEANUP_EXPR, \
			CLEANUP_POINT_EXPR, PLACEHOLDER_EXPR, \
			PLUS_EXPR, MINUS_EXPR, MULT_EXPR, TRUNC_DIV_EXPR, CEIL_DIV_EXPR, FLOOR_DIV_EXPR, ROUND_DIV_EXPR, TRUNC_MOD_EXPR, CEIL_MOD_EXPR, FLOOR_MOD_EXPR, ROUND_MOD_EXPR, RDIV_EXPR, \
			EXACT_DIV_EXPR, \
			FIX_TRUNC_EXPR, FIX_CEIL_EXPR, FIX_FLOOR_EXPR, FIX_ROUND_EXPR, \
			FLOAT_EXPR, NEGATE_EXPR, MIN_EXPR, MAX_EXPR, ABS_EXPR, \
			LSHIFT_EXPR, RSHIFT_EXPR, LROTATE_EXPR, RROTATE_EXPR, \
			BIT_IOR_EXPR, BIT_XOR_EXPR, BIT_AND_EXPR, BIT_NOT_EXPR, \
			TRUTH_ANDIF_EXPR, TRUTH_ORIF_EXPR, TRUTH_AND_EXPR, TRUTH_OR_EXPR, TRUTH_XOR_EXPR, TRUTH_NOT_EXPR, \
			LT_EXPR, LE_EXPR, GT_EXPR, GE_EXPR, EQ_EXPR, NE_EXPR, \
			UNORDERED_EXPR, ORDERED_EXPR, \
			UNLT_EXPR, UNLE_EXPR, UNGT_EXPR, UNGE_EXPR, UNEQ_EXPR, \
			LTGT_EXPR, \
			RANGE_EXPR, CONVERT_EXPR, NOP_EXPR, NON_LVALUE_EXPR, VIEW_CONVERT_EXPR, SAVE_EXPR, ADDR_EXPR, FDESC_EXPR, FDESC_EXPR, COMPLEX_EXPR, CONJ_EXPR, \
			REALPART_EXPR, IMAGPART_EXPR, \
			PREDECREMENT_EXPR, PREINCREMENT_EXPR, POSTDECREMENT_EXPR, POSTINCREMENT_EXPR, \
			VA_ARG_EXPR, TRY_CATCH_EXPR, TRY_FINALLY_EXPR, DECL_EXPR, LABEL_EXPR, GOTO_EXPR, RETURN_EXPR, EXIT_EXPR, LOOP_EXPR, \
			SWITCH_EXPR, CASE_LABEL_EXPR, \
			RESX_EXPR, ASM_EXPR, CATCH_EXPR, EH_FILTER_EXPR, ASSERT_EXPR, WITH_SIZE_EXPR, REALIGN_LOAD_EXPR, TARGET_MEM_REF, \
			REDUC_MAX_EXPR, REDUC_MIN_EXPR, REDUC_PLUS_EXPR, \
			VEC_LSHIFT_EXPR, VEC_RSHIFT_EXPR

#define ALL_TYPES	OFFSET_TYPE, ENUMERAL_TYPE, BOOLEAN_TYPE, CHAR_TYPE, INTEGER_TYPE, REAL_TYPE, POINTER_TYPE, REFERENCE_TYPE, COMPLEX_TYPE, VECTOR_TYPE, ARRAY_TYPE, RECORD_TYPE, UNION_TYPE, \
			QUAL_UNION_TYPE, VOID_TYPE, FUNCTION_TYPE, METHOD_TYPE, LANG_TYPE

#define ALL_CSTS	INTEGER_CST, REAL_CST, COMPLEX_CST, VECTOR_CST, STRING_CST

#define ALL_DECLS	FUNCTION_DECL, LABEL_DECL, FIELD_DECL, VAR_DECL, CONST_DECL, PARM_DECL, TYPE_DECL, RESULT_DECL, NAMESPACE_DECL, TRANSLATION_UNIT_DECL

#define RDONLY	1
#define RDWR	2

DEFTREEPARAMETER(identifier_length,		RDONLY,	SIZE_T,			IDENTIFIER_LENGTH,	IDENTIFIER_NODE)
DEFTREEPARAMETER(identifier_pointer,		RDONLY,	STRING,			IDENTIFIER_POINTER,	IDENTIFIER_NODE)
DEFTREEPARAMETER(tree_chain,			RDWR,	TREE,			TREE_CHAIN,		TREE_LIST, ENUMERAL_TYPE, RECORD_TYPE, UNION_TYPE, QUAL_UNION_TYPE)
DEFTREEPARAMETER(tree_value,			RDWR,	TREE,			TREE_VALUE,		TREE_LIST)
DEFTREEPARAMETER(tree_purpose,			RDWR,	TREE,			TREE_PURPOSE,		TREE_LIST)
DEFTREEPARAMETER(tree_vec_length,		RDWR,	SIZE_T,			TREE_VEC_LENGTH,	TREE_VEC)
DEFTREEPARAM_VECTOR(tree_vec_elts,		RDWR,	TREE,	TREE_VEC_ELT,	TREE_VEC_LENGTH,	TREE_VEC)
DEFTREEPARAMETER(block_vars,			RDWR,	TREE,			BLOCK_VARS,		BLOCK)
DEFTREEPARAMETER(block_chain,			RDWR,	TREE,			BLOCK_CHAIN,		BLOCK)
DEFTREEPARAMETER(block_abstract_origin,		RDWR,	TREE,			BLOCK_ABSTRACT_ORIGIN,	BLOCK)
DEFTREEPARAMETER(block_abstract,		RDWR,	BOOL,			BLOCK_ABSTRACT,		BLOCK)
DEFTREEPARAMETER(tree_asm_written,		RDWR,	BOOL,			TREE_ASM_WRITTEN,	BLOCK)
DEFTREEPARAMETER(type_size,			RDWR,	TREE,			TYPE_SIZE,		ALL_TYPES)
DEFTREEPARAMETER(type_mode,			RDWR,	MACHINE_MODE,		TYPE_MODE,		ALL_TYPES)
DEFTREEPARAMETER(type_pointer_to,		RDWR,	TREE,			TYPE_POINTER_TO,	ALL_TYPES)
DEFTREEPARAMETER(type_next_variant,		RDWR,	TREE,			TYPE_NEXT_VARIANT,	ALL_TYPES)
DEFTREEPARAMETER(type_main_variant,		RDWR,	TREE,			TYPE_MAIN_VARIANT,	ALL_TYPES)
DEFTREEPARAMETER(type_name,			RDWR,	TREE,			TYPE_NAME,		ALL_TYPES)
DEFTREEPARAMETER(type_context,			RDWR,	TREE,			TYPE_CONTEXT,		ALL_TYPES)
DEFTREEPARAMETER(type_offset_basetype,		RDWR,	TREE,			TYPE_OFFSET_BASETYPE,	OFFSET_TYPE)
DEFTREEPARAMETER(type_values,			RDWR,	TREE,			TYPE_VALUES,		ENUMERAL_TYPE)
DEFTREEPARAMETER(type_min_value,		RDWR,	TREE,			TYPE_MIN_VALUE,		INTEGER_TYPE)
DEFTREEPARAMETER(type_max_value,		RDWR,	TREE,			TYPE_MAX_VALUE,		INTEGER_TYPE)
DEFTREEPARAMETER(type_precision,		RDWR,	SIZE_T,			TYPE_PRECISION,		INTEGER_TYPE, REAL_TYPE, VECTOR_TYPE)
DEFTREEPARAMETER(tree_type,			RDWR,	TREE,			TREE_TYPE,		INTEGER_TYPE, POINTER_TYPE, REFERENCE_TYPE, COMPLEX_TYPE, VECTOR_TYPE, ALL_CSTS, FUNCTION_DECL, \
													FIELD_DECL, VAR_DECL, CONST_DECL, PARM_DECL, TYPE_DECL, RESULT_DECL, NAMESPACE_DECL, \
													TRANSLATION_UNIT_DECL, ALL_EXPRS)
DEFTREEPARAMETER(type_domain,			RDWR,	TREE,			TYPE_DOMAIN,		ARRAY_TYPE)
DEFTREEPARAMETER(type_fields,			RDWR,	TREE,			TYPE_FIELDS,		RECORD_TYPE, UNION_TYPE, QUAL_UNION_TYPE)
DEFTREEPARAMETER(type_method_basetype,		RDWR,	TREE,			TYPE_METHOD_BASETYPE,	METHOD_TYPE)
DEFTREEPARAMETER(type_arg_types,		RDWR,	TREE,			TYPE_ARG_TYPES,		METHOD_TYPE)
DEFTREEPARAMETER(tree_int_cst_low,		RDWR,	INT_CST_LOW,		TREE_INT_CST_LOW,	INTEGER_CST)
DEFTREEPARAMETER(tree_int_cst_high,		RDWR,	INT_CST_HIGH,		TREE_INT_CST_HIGH,	INTEGER_CST)
DEFTREEPARAMETER(tree_real_cst,			RDWR,	REAL,			TREE_REAL_CST,		REAL_CST)
DEFTREEPARAMETER(tree_realpart,			RDWR,	TREE,			TREE_REALPART,		COMPLEX_CST)
DEFTREEPARAMETER(tree_imagpart,			RDWR,	TREE,			TREE_IMAGPART,		COMPLEX_CST)
DEFTREEPARAMETER(tree_vector_cst_elts,		RDWR,	TREE,			TREE_VECTOR_CST_ELTS,	VECTOR_CST)
DEFTREEPARAMETER(tree_string_length,		RDONLY,	SIZE_T,			TREE_STRING_LENGTH,	STRING_CST)
DEFTREEPARAMETER(tree_string_pointer,		RDONLY,	STRING,			TREE_STRING_POINTER,	STRING_CST)
DEFTREEPARAMETER(decl_name,			RDWR,	TREE,			DECL_NAME,		ALL_DECLS)
DEFTREEPARAMETER(decl_context,			RDWR,	TREE,			DECL_CONTEXT,		ALL_DECLS)
DEFTREEPARAMETER(decl_abstract_origin,		RDWR,	TREE,			DECL_ABSTRACT_ORIGIN,	ALL_DECLS)
DEFTREEPARAMETER(decl_align,			RDWR,	SIZE_T,			DECL_ALIGN,		FUNCTION_DECL, FIELD_DECL, VAR_DECL, PARM_DECL, RESULT_DECL, NAMESPACE_DECL, TRANSLATION_UNIT_DECL)
DEFTREEPARAMETER(decl_size,			RDWR,	TREE,			DECL_SIZE,		FUNCTION_DECL, FIELD_DECL, VAR_DECL, PARM_DECL, RESULT_DECL, NAMESPACE_DECL, TRANSLATION_UNIT_DECL)
DEFTREEPARAMETER(decl_mode,			RDWR,	MACHINE_MODE,		DECL_MODE,		FUNCTION_DECL, FIELD_DECL, VAR_DECL, PARM_DECL, RESULT_DECL, NAMESPACE_DECL, TRANSLATION_UNIT_DECL)
DEFTREEPARAMETER(decl_field_bit_offset,		RDWR,	TREE,			DECL_FIELD_BIT_OFFSET,	FIELD_DECL, PARM_DECL)
DEFTREEPARAMETER(decl_initial,			RDWR,	TREE,			DECL_INITIAL,		VAR_DECL, CONST_DECL, FUNCTION_DECL, LABEL_DECL)
DEFTREEPARAMETER(decl_arg_type,			RDWR,	TREE,			DECL_ARG_TYPE,		PARM_DECL)
DEFTREEPARAMETER(decl_arguments,		RDWR,	TREE,			DECL_ARGUMENTS,		FUNCTION_DECL)
DEFTREEPARAMETER(decl_result,			RDWR,	TREE,			DECL_RESULT,		FUNCTION_DECL)
DEFTREEPARAMETER(decl_function_code,		RDWR,	BUILT_IN_FUNCTION,	DECL_FUNCTION_CODE,	FUNCTION_DECL)
DEFTREEPARAMETER(decl_source_file,		RDONLY,	STRING,			DECL_SOURCE_FILE,	ALL_DECLS)
DEFTREEPARAMETER(decl_source_line,		RDONLY,	OFF_T,			DECL_SOURCE_LINE,	ALL_DECLS)
DEFTREEPARAMETER(decl_abstract,			RDWR,	BOOL,			DECL_ABSTRACT,		ALL_DECLS)
DEFTREEPARAMETER(bit_field_ref_unsigned,	RDWR,	BOOL,			BIT_FIELD_REF_UNSIGNED,	BIT_FIELD_REF)
DEFTREEPARAMETER(tree_addressable,		RDWR,	BOOL,			TREE_ADDRESSABLE,	VAR_DECL, VIEW_CONVERT_EXPR)
DEFTREEPARAMETER(phi_result,			RDONLY,	TREE,			PHI_RESULT,		PHI_NODE)
DEFTREEPARAMETER(phi_arg_length,		RDONLY,	SIZE_T,			PHI_ARG_LENGTH,		PHI_NODE)
DEFTREEPARAM_VECTOR(phi_arg_elts,		RDONLY,	PHI_ARG, PHI_ARG_ELT,	PHI_ARG_LENGTH,		PHI_NODE)

#undef ALL_EXPRS
#undef ALL_TYPES
#undef ALL_CSTS
#undef ALL_DECLS 
\end{verbatim}
	
\subsection{Semantics.}
The match semantics are such that a list of patterns must be matched in
sequence, the sequence being any ordered possible traversal along the control
flow graph of the program. We define a language we use to specify a match: 
G        GIMPLE Object
G'       GIMPLE Object
G F G'   G is Followed by G' sometime in the future (G' is a possible successor
         to G in the CFG)
G I G'   G is Immediately followed by G' (G' is the immediate successor to G in
         the CFG)
G#Cond   G that obeys a condition Cond. Cond may contain not, and, or. 
G|G'     G or G'

Thus, $c=a+b$ becomes 
   $(\$temp=a+b I c=\$temp)|(\$temp=a+b F \$X#( not Assign(\$temp,\$ANY)) F c=\$temp)$.

An expression in the above languages is sent to the GQE to be evaluated, with
variable bindings being sent out of band.



\section{Rule Application Module}
\TODO{Add Sean's description of modification engine here.}
We apply rules by modifying GIMPLE code directly, via the GCC plugin. It
exports certain functions, such as:
\begin{list}
	\item $apply_before(location,code)$
	\item $apply_after(location,code)$
	\item $apply_instead(location,code)$
\end{list}

These functions force recompilation of the involved GIMPLE function, and the
update of all analysis results. The efficient implementation of that requires
that all analyses involved be incremental, in some way. This section describes
the details.

\paragraph{Making CFG generation incremental.} When code changes, we update the
underlying CFG. When no function pointers are involved, this is trivial to do.
If function pointers are involved, type analysis is used to restrict possible
target functions. Using this information, and labels on aliasing workstation
updates, the alias analysis results are recomputed. After this, the CFG is
re-updated again, and the final alias results are computed.

\paragraph{Making Choi's analysis incremental.} We follow in the steps of Hind
\cite{hind01} in making Choi's analysis use a worklist. We then continue by
adapting Goyal's work \cite{goyal02}, which cements the use of a work-set
algorithm. Our modification, as described in the previous section, primarily
involve the use of pre- and post-function information, as we restrict the
analysis to the insides of a single function.

The incrementalization comes from labelling all modifications with change
labels. Change labels specify precisely what the current modification to the
worklist depends on. Thus, when a code-modifying event happens, locate all
modifications to the worklist that depended on the code that was modified, and
we propagate the change by recomputing only what is necessary. Unfortunately,
even minor changes in variable assignments may significantly change may-alias
sets. For examples, assuming may-alias sets (a1,a2,a3) and (a4,a5,a6), the
addition of the alias pair (a1,a4) fundamentally changes the may-alias sets to
(a1,a2,a3,a4,a5,a6). Thus, it is impossible to state that incrementalization is
an asymptotic win in all cases. In practice, though, this approach provides
very significant speedups. \TODO{Experiments.} 

\paragraph{Incrementalizing Interprocedural Alias Analysis.} 
\TODO{Add this, after implementation.}


\section{Parser for C Patterns.} 
The parser is based on ELSA. Its main goal is to take a C expression, and to
generate a (preferably typed) AST that can be converted into a query for
\query. As the argument to query is not a GIMPLE AST, but an object resembling
the AST, and as GCC cannot even parse truly incomplete code, the parser is
independent of GCC, and generates its own AST and typing rules. 

\paragraph{Accepted Grammar.}
The accepted grammar is the ISO C99 grammar, with the following extension:
\begin{verbatim}
Any identifier or expression may be replaced with $E, where E is a valid
identifier name in ISO C99.

Multiple statements may be replaced by a single $E.
\end{verbatim}

\paragraph{Type Inference.} Typing is performed according to C typing rules,
and infers concrete types, where possible. If concrete types are impossible to
infer, but the type equations are not degenerate, the parser infers typing
rules via deduced types.  For example, a.x+1 infers that
a.x<int|long|float|...> where ... is all the types to which one can add a one.
Further, elaboration is done where possible, thus, types are not just a<int>,
but a<int[1,2]> , which means that a is an int with values 1 and 2. Clearly, it
is not always possible to infer such things, and at that point, the parser
drops back to inferring just types. 

Type equations are represented as a constraint system, and solved according to
the work of Wang and Smith, \cite{Smith01ecoop}, though simplified, as C
contains no explicit subtyping. Pointers are treated as references, with the
given type as a concrete type, and void being the root type of all types. The
algorithm is Agesen's Cartesian Product Algorithm (CPA).

***********************************************************************
END OF VALID DOCUMENT
***********************************************************************
Conversion from C patterns to patterns:
1. Patterns are evaluated by ELSA, with alternatives being generated. 
2. Using backtracking, all complete versions of patterns (w/out alternatives) are generated.
For each complete (w/out alternatives) pattern:
    3. Typecalling is performed on patterns (i.e., type equations are written up for the pattern)
    4. Type equations are solved, to the least genericity possible.
    5. Solutions of type equations are used to generate conditions on all appropriate nodes in the pattern dict.
    6. The _if clause_ is converted to conjunctive form, and each element of the conjunction is added to all appropriate nodes in the pattern dict.

7. All pattern dicts are merged, with alternates introduced where they diverge.

This results in a complete pattern dict that can be sent to GCC.


InvTS - Query Generation Module	 
	This AST is then converted into the query form by a slicing procedure, 
	followed by a transformation done via a traversal. The slicing procedure 
	removes the parts of the generated AST that are irrelevant to the query, 
	but needed for typing and semantic elaboration.	The transformation is just 
	a single pass that renames and reorders AST nodes to better match what can 
	be easily queried over GIMPLE ASTs. Metavariables are replaced by wildcards 
	with bindings, and ambiguities are expanded into multiple queries.
	
InvTS - Inter-invocation module (via database)
	The need for an inter-invocation module is exemplified by the following 
	compilation procedure (pseudo-code):
		gcc a.c -output a.o
		gcc b.c -output b.o
		link a.o b.o
	This means that during the invocation of InvTS for a.c, it does not know 
	about b.c.
	This presents a problem if an at clause for an inv matched in a.c is matched 
	in b.c.
	The solution is to maintain state in the database, and require 2 invocations
	of InvTS. The first invocation computes all the inv sites (and the bindings 
	at them), and the 2nd invocation then actually computes the at sites, and 
	performs the transformation. 
	The required magic to get make to understand this is still to in 
	development. The main idea, though, is to set the date of the to file after 
	the first invocation to the past, so that make will be forced to rebuild it 
	for the 2nd invocation.
	
InvTS - InvTL LM
	The LM handles the interaction between the GCC Python plugin and InvTL. It 
	is implemented according to the spec in the docs (Manual/Python LM.pdf).


================================================================================
================================================================================
Matching pattern dicts in GCC (Overall):
Nodes are recursively matched against the GIMPLE tree.
Each node is considered it's own stackframe, with the local space being the bound metavariables.

Each type of possible key in a dict is considered an op, which is executed with the parameter being its value. Most of these ops are just simple recursive matching. Some special ones worth mentioning include:
name 
    This node matches either a literal, of if prefixed with "$", a metavariable.
    The match of a metavariable depends on whether it is fully bound or not.
    If it is fully bound, a pattern dict is generated based on the binding site, and matched.
    If not, a new biding site is created in the stackframe, and the variable is stored in the local space.
condition
    A condition is evaluated by passing it to the Python GCC LM, where it is converted to Python bytecode, and evaluated. 
aternative
    A simple backtracking algorithm is used to match alternates within alternates.
    
    



